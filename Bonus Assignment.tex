%LaTeX reported generated by EES
\documentclass[10pt,fleqn]{article}
\mathindent 0.0in
\usepackage[ansinew]{inputenc}
\usepackage{times}
\usepackage{graphicx}
\usepackage{color}
\usepackage{textcomp}
\definecolor{silver}{rgb}{0.75,0.75,0.75}
\definecolor{gray}{rgb}{0.5,0.5,0.5}
\definecolor{aqua}{rgb}{0.5,1,1}
\definecolor{navy}{rgb}{0.0,0.0,0.5}
\definecolor{orange}{rgb}{1.0,0.5,0.0}
\definecolor{teal}{rgb}{0.25,0.5,0.5}
\definecolor{olive}{rgb}{0.5,0.5,0.0}
\definecolor{purple}{rgb}{0.5,0.0,0.5}
\definecolor{brown}{rgb}{0.5,0.25,0.0}
\definecolor{fuchsia}{rgb}{1.0,0.5,1.0}
\definecolor{buff}{rgb}{1.0,0.94,0.80}
\definecolor{lime}{rgb}{0.5,1.0,0.0}
\setlength{\headsep}{-0.6in}
\setlength{\textheight}{9in}
\setlength{\footskip}{1.0 in}
\setlength{\oddsidemargin}{-0.2in}
\setlength{\evensidemargin}{-0.2in}
\setlength{\textwidth}{6.8in}
\usepackage{longtable}
\def\headline#1{\hbox to \hsize{\hrulefill\quad\lower .3em\hbox{#1}\quad\hrulefill}}
\newcommand{\abs}[1]{\left|#1\right|}
\newcommand{\F}[1]{\mbox{$#1$}}
\newcommand{\K}[1]{\mbox{\sf#1\ \ \mit}}
\newcommand{\KS}[1]{\mbox{\sf\ \ #1\ \ \mit}}
\newcommand{\SC}[1]{\mbox{`#1'}\  }
\newcommand{\V}[1]{\mbox{$ #1 $}}
\newcommand{\I}{\mbox{\hspace{0.20in}}}
\newcommand{\temperature}{\mathrm{T}}
\newcommand{\pressure}{\mathrm{P}}
\newcommand{\volume}{\mathrm{v}}
\newcommand{\density}{\mathrm{\rho}}
\newcommand{\intenergy}{\mathrm{u}}
\newcommand{\enthalpy}{\mathrm{h}}
\newcommand{\entropy}{\mathrm{s}}
\newcommand{\molarmass}{\mathrm{MW}}
\newcommand{\enthalpyfusion}{\mathrm{\Delta h_{fusion}}}
\newcommand{\quality}{\mathrm{x}}
\newcommand{\viscosity}{\mathrm{\mu}}
\newcommand{\conductivity}{\mathrm{k}}
\newcommand{\prandtl}{\mathrm{P_r}}
\newcommand{\cp}{\mathrm{c_p}}
\newcommand{\cv}{\mathrm{c_v}}
\newcommand{\specheat}{\mathrm{c_p}}
\newcommand{\soundspeed}{\mathrm{c}}
\newcommand{\wetbulb}{\mathrm{wb}}
\newcommand{\humrat}{\mathrm{\omega}}
\newcommand{\acentricfactor}{\mathrm{\omega}}
\newcommand{\relhum}{\mathrm{\phi}}
\newcommand{\dewpoint}{\mathrm{DP}}
\newcommand{\volexpcoef}{\mathrm{\beta}}
\newcommand{\compressibilityfactor}{\mathrm{Z}}
\newcommand{\surfacetension}{\mathrm{\gamma}}
\newcommand{\tcrit}{\mathrm{T_{crit}}}
\newcommand{\pcrit}{\mathrm{P_{crit}}}
\newcommand{\vcrit}{\mathrm{v_{crit}}}
\newcommand{\ttriple}{\mathrm{T_{triple}}}
\newcommand{\fugacity}{\mathrm{fugacity}}
\newcommand{\tsat}{\mathrm{T_{sat}}}
\newcommand{\psat}{\mathrm{P_{sat}}}
\newcommand{\eklj}{\mathrm{ek_{LJ}}}
\newcommand{\sigmalj}{\mathrm{\sigma_{LJ}}}
\newcommand{\isentropicexponent}{\mathrm{k_{s}}}
\newcommand{\thermaldiffusivity}{\mathrm{\alpha}}
\newcommand{\kinematicviscosity}{\mathrm{\nu}}
\newcommand{\isothermalcompress}{\mathrm{K_{T}}}
\newcommand{\henryconstantwater}{\mathrm{HenryConst}}
\begin{document}
\begin{center}
\bf \mbox{BONUS ASSIGNMENT}
\vspace{0.2 in}
\end{center}
\subsection*{Equations}

\vspace{0.04in}
\noindent
\rm Please make sure your �external$_{flow.lib�}$ EES library includes both �external$_{flow,sphere�}$ AND �external$_{flow,sphere,nd�}$ procedures

\vspace{0.04in}
\noindent
\rm Specifications for the ball
\begin{equation}
\label{EES Eqn:1}
m = 0.1   \   \left[ \rm kg \right] 
\end{equation}
\rm
\begin{equation}
\label{EES Eqn:2}
D = 0.3   \   \left[ \rm m \right] 
\end{equation}
\rm
\begin{equation}
\label{EES Eqn:3}
A = \pi \cdot  \frac {D^{2}}{ 4 } 
\end{equation}
\begin{equation}
\label{EES Eqn:4}
V = \pi \cdot  \frac {D^{3}}{ 6 } 
\end{equation}

\vspace{0.04in}
\noindent
\rm Our shell is hollow on the inside, so we cannot obtain its density using EES Thermophysical property functions
\begin{equation}
\label{EES Eqn:5}
\rho_{ball} = m/V 
\end{equation}

\vspace{0.04in}
\noindent
\rm Temperatures
\begin{equation}
\label{EES Eqn:6}
T_{inf} = 17   \   \left[ \rm C \right] 
\end{equation}
\rm
\begin{equation}
\label{EES Eqn:7}
T_{initial} = 250   \   \left[ \rm C \right] 
\end{equation}
\rm

\vspace{0.04in}
\noindent
\rm In previous iterations of this code, I thought It would be wise to consider the density of air and the specific heat of copper to be variable with temperature to improve the accuracy of my final answer. during testing I realized that calculating each of these two in every step of my time integral would only slightly alter my final answer (from 18.98 Celsius to 19.38 Celsius) but would more than triple the processing time. So I went back to using average values for them
\begin{equation}
\label{EES Eqn:8}
T_{s,ave} = \frac { \left( T_{initial} + 18.98   \   \left[ \rm C \right] \right) }{ 2	 } 
\mbox{\I I acquired T$_{final}$ = 18.98 from previous runs of the code and iterated from there}
\end{equation}
\begin{equation}
\label{EES Eqn:9}
T_{film} = \frac { \left( T_{inf} + T_{s,ave} \right) }{ 2	 } 
\end{equation}
\begin{equation}
\label{EES Eqn:10}
\rho_{air} = \density \left(\F{Air}_{ha},\mbox{\ T}=T_{film},\mbox{\ P}=101.3   \   \left[ \rm kPa \right] \right)  
\end{equation}
\rm
\begin{equation}
\label{EES Eqn:11}
C_{p} = \cp \left(\F{Copper},\mbox{\ T}=T_{s,ave} \right)  
\end{equation}

\vspace{0.04in}
\noindent
\rm Initial values for remaining height, surface temperature and speed
\begin{equation}
\label{EES Eqn:12}
\V{RemainingHeight} _{1} = 100   \   \left[ \rm m \right] + 87   \   \left[ \rm m \right] 
\end{equation}
\rm
\begin{equation}
\label{EES Eqn:13}
T_{s,1} = T_{initial} 
\end{equation}
\begin{equation}
\label{EES Eqn:14}
U_{1} = 0   \   \left[ \rm m/s \right] 
\end{equation}
\rm

\vspace{0.04in}
\noindent
\rm Our chosen time step for discretization of the time integral
\begin{equation}
\label{EES Eqn:15}
{\delta }_{t} = 0.04   \   \left[ \rm s \right] 
\end{equation}
\rm

\vspace{0.04in}
\noindent
\rm I set our integrations higher bound (673 * 0.04 seconds ~ 27 seconds) to the exact time it takes for the ball to reach the ground. In other words, RemainingHeight[674] is the first negative value of RemainingHeight[]
\begin{equation}
\label{EES Eqn:16}
\K{duplicate} i=1,\ 673 
\end{equation}

\vspace{0.04in}
\noindent
\rm Using the EES function for external convective cooling over a smooth sphere, we are only interested in the drag force F$_{D,] and convective heat transfer coefficient h,]}$
\begin{equation}
\I \label{EES Eqn:17}
\K{call} \F{external_{flow,sphere}}{ \left( \SC{Air\_ha},\ T_{inf},\ T_{s,i},\ 101.3   \   \left[ \rm kPa \right],\ U_{i},\ D: F_{D,i},\ h_{i},\ C_{D,i},\ \V{Nusselt} _{i},\ \V{Re} _{i} \right) } 
\end{equation}
\rm

\vspace{0.04in}
\noindent
\rm Finding the Instantaneous acceleration using F$_{net}$ = ma     and     F$_{net}$ = W - F$_{B}$ - F$_{D}$
\begin{equation}
\I \label{EES Eqn:18}
m \cdot  a_{i} = V \cdot  9.806   \   \left[ \rm m/s^{2} \right] \cdot   \left( \rho_{ball} - \rho_{air} \right)  - F_{D,i} 
\end{equation}
\rm
\begin{equation}
\I \label{EES Eqn:19}
\V{RemainingHeight} _{i+1} = \V{RemainingHeight} _{i} -  \left( U_{i} \cdot  {\delta }_{t} \right)  -  \left( 0.5 \cdot  a_{i} \cdot  {\delta }_{t}^{2} \right)  
\mbox{\I delta x = Vt + (1/2)at$^{2}$}
\end{equation}
\begin{equation}
\I \label{EES Eqn:20}
U_{i+1} = U_{i} +  \left( a_{i} \cdot  {\delta }_{t} \right) 	 
\mbox{\I delta V = at}
\end{equation}

\vspace{0.04in}
\noindent
\rm Using the lumped capacitance model. I will show later why it can be used to accurately model this problem
\begin{equation}
\I \label{EES Eqn:21}
T_{s,i+1} - T_{inf} =  \left( T_{s,i} - T_{inf} \right)  \cdot  \exp{ \left( -1 \cdot  h_{i} \cdot  4 \cdot  A \cdot  \frac {{\delta }_{t}}{  \left( m \cdot  C_{p} \right)  } \right) } 
\end{equation}
\begin{equation}
\label{EES Eqn:22}
\K{end} 
\end{equation}

\vspace{0.04in}
\noindent
\rm Finding the maximum Biot number value to see if using the lumped capacitance model is a good estimation for our problem or not. The Biot number in this problem amounts to only	       Bi = 0.00000231 which is (much) lower than Bi = 0.1 so lumped capacitance model is accurate\newline
In finding the Biot number, it would have been best to use the thermal resistance definition of the Biot number (Bi = R$_{cond}$ / R$_{conv}$) as we already know the radial thermal resistance of a hollow shell, but I still decided to use the L$_{c}$ method
\begin{equation}
\label{EES Eqn:23}
V_{real} = \frac {m}{ \density \left(\F{Copper},\mbox{\ T}=T_{final} \right)  } 
\end{equation}
\begin{equation}
\label{EES Eqn:24}
L_{c} = \frac {V_{real}}{  \left( 4 \cdot  A \right) 	 } 
\mbox{\I ~~~~~~~~~~~~~~~~~~~~~~~~~~~~~~~~~~~~~~~~~~~~~~~~~~~~~~~~~~~~~~~~~~~}
\end{equation}
\begin{equation}
\label{EES Eqn:25}
\V{Bi}  = h_{673} \cdot  \frac {L_{c}}{ \conductivity \left(\F{Copper},\mbox{\ T}=T_{final} \right) 	 } 
\mbox{\I ~~~~~~~~~~~~~~~~~~~~~~~~~~~~~~~~~~~~~~~~~~~~~~~~~~~~~~~~~~~~~~~~~~~}
\end{equation}

\vspace{0.04in}
\noindent
\rm Finding out T$_{s}$ when the ball reaches our hand. T$_{final}$ is equal to 18.98 degrees Celsius So ~~~~~~~~~~~~~~~~~~~~~The Ball Is Perfectly Safe to Catch~~~~~~~~~~~~~~~~~~~~~
\begin{equation}
\label{EES Eqn:26}
T_{final} = T_{s,673	} 
\mbox{\I ~~~~~~~~~~~~~~~~~~~~~~~~~~~~~~~~~~~~~~~~~~~~~~~~~~~~~~~~~~~~~~~~~~~}
\end{equation}

\vspace{0.04in}
\noindent
\rm ~~~~~~~~~~~~~~~~~~~~~~~~~~~~~~~~~~~~~~~~~~~~~~~~~~~~~~~~~~~~~~~~~~~

\vspace{0.04in}
\noindent
\rm If we didn't want to use the EES function for external convection, we could have instead used:\newline
A) The equation C$_{D}$ = [ 24  /  Re ]   +   [ (2.6*(Re / 5))  /  (1+(Re / 5)$^{1.52}$) ]   +   [ (0.411*(Re / 2.63*10$^{5}$)$^{-7.94}$)  /  (1+(Re / 2.63*10$^{5}$)$^{-8}$) ]   +   [ (0.25*(Re / 10$^{6}$))  /  (1+(Re / 10$^{6}$))]\newline
One advantage of this formula is that it can accurately predict the drag coefficients of flows with bigger than 2 * 10$^{5}$ Reynolds numbers (Turbulent flows)\newline
		Source: Faith A. Morrison, �Data Correlation for Drag Coefficient for Sphere,�\newline
		Department of Chemical Engineering, Michigan Technological University, Houghton, MI,\newline
		www.chem.mtu.edu/~fmorriso/DataCorrelationForSphereDrag2016.pdf\newline
B) the Whitaker correlation to find Nusselt's number as Nusselt = 2 + (0.4*Re$^{0.5}$  +  0.06*Re$^{0.667}$) * Pr$^{0.4}$ * ($\mu$/$\mu$$_{s}$)$^{0.25\newline}$
C) Re = U * D / kinematicviscosity(Air$_{ha}$,T=T$_{film}$, P=101.3 [kPa]) for finding Re\newline
D) F$_{D}$ = ($\rho$$_{air}$ * U$^{2}$ * A * C$_{D}$) / 2 to find drag force\newline
And the final segment would have looked something like:\newline
Pr = prandtl(Air$_{ha}$,T=T$_{inf}$, P=101.3 [kPa])\newline
nu = kinematicviscosity(Air$_{ha}$,T=T$_{film}$, P=101.3 [kPa])\newline
mu = viscosity(Air$_{ha}$,T=T$_{inf}$, P=101.3 [kPa])\newline
mu$_{s}$ = viscosity(Air$_{ha}$,T=T$_{film}$, P=101.3 [kPa])\newline
k$_{f}$ = conductivity(Air$_{ha}$,T=T$_{film}$, P=101.3 [kPa])\newline
Duplicate i=1,673\newline
	Re[i] = U[i] * D / $\nu$\newline
	C$_{D,i}$ = ( 24  /  Re[i] )   +   ( (2.6*(Re[i] / 5))  /  (1+(Re[i] / 5)$^{1.52}$) )   +   ( (0.411*(Re[i] / 2.63*10$^{5}$)$^{-7.94}$)  /  (1+(Re[i] / 2.63*10$^{5}$)$^{-8}$) )   +   ( (0.25*(Re[i] / 10$^{6}$))  /  (1+(Re[i] / 10$^{6}$)))\newline
	F$_{D,i}$ = ($\rho$$_{air,i}$ * U[i]$^{2}$ * A * C$_{D,i}$) / 2\newline
	F$_{net,i}$ = V * 9.806 [m/s$^{2}$] * ($\rho$$_{ball}$ - $\rho$$_{air}$) - F$_{D,i}$\newline
	m * a[i] = F$_{net,i}$\newline
	RemainingHeight[i+1] = RemainingHeight[i] - (U[i] * ${\delta }$$_{t}$) - (0.5 * a[i] * ${\delta }$$_{t}$^{2}$) \newline
	U[i+1] = U[i] + (a[i] * ${\delta }$$_{t}$)\newline
	Nusselt[i] = 2 + (0.4*Re[i]$^{0.5}$  +  0.06*Re[i]$^{0.667}$) * Pr$^{0.4}$ * ($\mu$/$\mu$$_{s}$)$^{0.25\newline}$
	h[i] = Nusselt[i] * k$_{f}$ / D\newline
	T$_{s,i+1}$ - T$_{inf}$ = (T$_{s,i}$ - T$_{inf}$) * exp(-1 * h[i] * 4 * A * ${\delta }$$_{t}$ / m * C$_{p}$)\newline
End

\end{document}
